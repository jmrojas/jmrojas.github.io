\documentclass[a4paper]{article}

\usepackage[a4paper,left=25mm,top=20mm,right=25mm,bottom=15mm]{geometry}

\usepackage[compact]{titlesec}

\title{Teaching Statement\\{\Large Jos\'e Miguel Rojas}}
\date{}

\begin{document}

\maketitle

% Teaching philosophy
% 1. Teaching Philosophy
% - Learning by doing
% - Effective lecturing
% - Effective use of time outside contact hours


\section{Teaching philosophy}

\emph{Learning by doing} is a cornerstone of my teaching philosophy. I
strongly believe students make the most out of a lecture when instead
of sitting through a number of slides they actively engage in the
learning process, create and make things work with their own hands,
and reflect on their experiences. In my view, the role of an educator
largely involves enabling and fostering a learning environment for
students to discover knowledge and apply it by themselves. I am
committed to this approach, as it enables teaching to become more
effective and rewarding for the lecturer, and more importantly it
allows students to learn more effectively and more efficiently than
simply listening to lectures.

I believe that preparing lectures is as important as delivering them
and that their success highly depends on the effort put into planning
them. Understanding students needs and expectations from a course is
the first step to succeed. It is equally important to carefully
prepare the educational material to be used throughout a course, not
only material to be used during a lecture, but also material for
students to use their time effectively out of contact
hours. Well-selected literature, and carefully designed hands-on
exercises are key to maintaining students interest and engagement. My
assessment of students progress, therefore, takes into account the
amount and quality of work they do in the classroom and out of it,
too.

%Teaching assistant as an undergraduate student
\section{Teaching experience}

\subsection{Past experience}

I have ten years of teaching experience at different levels. As early
as my undergraduate years I engaged as a teaching assistant in core
modules of the computer science curriculum (Introduction to
Programming, Algorithms and Data Structures, and Compilers).
%
My strong inclination towards teaching lead me to enroll in a
postgraduate course in Higher Education right after finishing my
undergraduate (2007), where I gained theoretical knowledge and
practical skills on course design and planning, learning strategies,
teaching techniques and assessment methods.
%
Having an understanding of how people learn best has had a lasting
impact on how I deliver my lectures. I strive to apply what I learnt
in this postgraduate course every time I have an opportunity to teach
at university level and even when giving talks in seminars or academic
conferences.
%
For example, as a PhD student, I took a teaching assistant role in the
Data Structures module at the Technical University of Madrid (Spain),
where I was in charge of delivering lab sessions on object-oriented
programming and testing; selecting adequately-sized but challenging
problems was particularly decisive in that module to have students
commit to bringing their projects to completion.
%
My teaching experience at university level also includes a week-long
seminar on unit test generation at the Technical University of Madrid
(Spain), a masters module on software testing given at the Autonomous
University of Santa Cruz (Bolivia), and multiple short seminars and
invited talks given in Sheffield (UK), Madrid (Spain) and Santa Cruz
(Bolivia).


%Research on improving teaching
\subsection{Current activities}

\paragraph{Supervising students.} Interacting with students and
receiving and providing feedback is fundamental for effective
lecturing and learning. As a Research Associate in Software Testing at
the University of Sheffield, I have actively engaged with mentoring
undergraduate and masters students on small research
projects. Benefitting from several university funding schemes, I have
supervised three undergraduate students in the last two years. They
all completed their projects successfully, and one of them won the
Student Researcher of the Year award under the Sheffield Undergraduate
Research Experience (SURE)
program\footnote{http://www.sheffield.ac.uk/sure} and has since
decided to pursue a PhD degree after finishing his undergraduate at
the end of the current academic term. Out of these experiences, I am
convinced that one-on-one meetings are the most effective way to keep
students focused, to respond more adequately to their learning needs,
and to nurture their ability to independently develop their skills.

My engagement with student development can also be supported by my
close collaboration with six PhD students in my current department
over the past two years. From my postdoc role in the group, I enjoy
giving them support in technical aspects, feedback on their work, and
general advice on PhD-related matters. I try to drive my interactions
with them in terms of research quality and outcome. My collaboration
with all these PhD students has resulted in seven publications in top
international conferences or journals. Working closely with a diverse
set of students has also lead me to improved management skills such as
leadership, coordination, time management, delegation, planning, and
patience.

%Training students and professionals for empirical studies
\paragraph{Research on teaching software testing.} My main research
expertise is on software testing, and I believe reasoning about
program correctness, understanding the benefits of software testing,
and applying software testing techniques effectively is fundamental
for software engineers to develop high-quality software. Indeed,
leading software companies (e.g., Google) nowadays demand their
software developers to thoroughly test their code before deploying to
end users.  Unfortunately, software testing is often a neglected topic
in computer science/software engineering curricula, and is also
perceived as a dull, uninteresting task by students and
practitioners. I believe this needs to change and software testing
should be made an integral component of software engineering
education. As a result of these reflections, I recently started
working on gamification of software testing in order to emphasise the
importance of software quality and to present testing as a more
enjoyable activity. I lead the development of Code
Defenders\footnote{http://code-defenders.org}, an online game where
students can learn software testing concepts and techniques while
having fun\footnote{Paper accepted and to be presented at ICSE
  2017}. Preliminary experience suggests this approach can be
integrated into programming modules; for example, it has already been
used in software engineering lectures at George Mason University
(USA). I would be delighted to put this approach in practice when
lecturing at UCL.

%Teaching kids
\paragraph{Early-stage education.}  I am also a strong believer in
early-stage education. As educators, we have the responsibility to
inspire the next generations to get excited about computers and must
prepare them for a future where computers will be involved in most
activities, from recreational to life-critical. With this in mind, I
became an active volunteer of the Code Club network. I teach children
aged 8-11 to code in an after school coding club. Using the
Scratch\footnote{https://scratch.mit.edu} block-coding platform and
the BBC micro:bit\footnote{http://microbit.org}, I have witnessed how
children are capable of thinking logically to create and improve
non-trivial programs, all while having fun. Code Club is a nation-wide
initiative, and I plan to keep volunteering in the project (there are
at least 100 clubs within five miles from Gower St).

\section{Future at UCL}

As a Lecturer in Software Engineering at UCL, I wish to teach modules
related to software engineering, software quality and software
testing. I would be prepared to teach modules at any level
(undergraduate and masters). After a close look at the departamental
webpages, some of the modules I would be willing to teach any modules
related to software engineering. Out of those already offered by UCL,
and based on my previous experience and expertise, I would
particularly be interested in COMP101P-Principles of Programming,
COMP103P-Object Oriented Programming, and 203P-Software Engineering
and Human Computer Interaction, and COMPM023-Validation and
Verification, COMP207P-Compilers, and COMPM024-Tools and
Environments.

I believe my teaching philosophy and experience fit well with the
expectations for the post of Lecturer in Software Engineering at
UCL. I am enthusiastic about the opportunity to teach in one of the
leading schools in computer science in the UK. I am committed to
delivering effective and engaging lectures, and I am excited about the
idea of supervising UCL undergraduate, masters and PhD students at UCL and
look forward to inspiring the next generation of software engineers
and researchers.

\end{document}
