% -*- TeX-PDF-mode: t; TeX-master: "cv-jmrojas-en-ucl"; -*-

\documentclass[booktabs,nologo,helvetica,openbib,totpages]{europecv}
\usepackage[right]{eurosym}
\usepackage[T1]{fontenc}
\usepackage{graphicx}
\usepackage[a4paper,top=1.27cm,left=1cm,right=1cm,bottom=2cm]{geometry}
\usepackage[english]{babel}
\usepackage{bibentry}
\usepackage{url}
\renewcommand*{\ecvtitle}{\Large\textbf{Curriculum Vitae}}
\renewcommand{\ttdefault}{phv} % Uses Helvetica instead of fixed width font
\ecvname{Rojas, Jos\'{e} Miguel}
\ecvfootername{Jos\'{e} Miguel Rojas}
\tabularnewline
\ecvnationality{Bolivian}
\ecvdateofbirth{13 January 1984}
%\ecvaddress{Calle Jardines 9, 2B, 28013, Madrid, Spain}
\ecvtelephone[]{(+44) 7477537139}
%\ecvfax{(+44) 114 222 1810}
\ecvemail{\url{jmrs84@gmail.com}, \url{j.rojas@sheffield.ac.uk}}
\ecvscholar{\url{https://scholar.google.co.uk/citations?user=NeuqUtcAAAAJ}}
\ecvdblp{\url{http://dblp.uni-trier.de/pers/hd/r/Rojas:Jos=eacute=_Miguel}}
%\ecvgender{Male}
%\ecvpicture[width=2cm]{mypicture}
\ecvfootnote{}

\begin{document}
\selectlanguage{english}

\begin{europecv}
\ecvpersonalinfo[5pt]
% \ecvitem{\large\textbf{Desired employment/ Occupational~field}}{\large\textbf{(Remove if not relevant)}}

\ecvsection{Education}

% \ecvitem{Dates}{10/12/2007}
% \ecvitem{Title of qualification awarded}{\ldots}
% \ecvitem{Principal subjects/Occupational skills covered}{\ldots}
% \ecvitem{Name and type of organization providing education and training}{\ldots}
% \ecvitem{Level in national or international classification\footnote{If appropriate.}}{\ldots}

\ecvitem{Dates}{10/2009 -- 12/2013}
\ecvitem{Institution}{Technical University of Madrid, Spain}
\ecvitem{Degree awarded}{\emph{Phd. in Software and Systems}}
\ecvitem{Thesis title}{Test Case Generation in Object Oriented Programming}
\\
\ecvitem{Dates}{02/2008 -- 09/2009}
\ecvitem{Institution}{Autonomous University of Madrid, Spain}
\ecvitem{Study}{\emph{Postgraduate Study in Computer
  Engineering and Telecommunications}}
\\
\ecvitem{Dates}{04/2005 -- 12/2005}
\ecvitem{Institution}{Autonomous University of Santa Cruz, Bolivia}
\ecvitem{Qualification awarded}{\emph{Postgraduate Diploma in Higher Education}}

%\\
%\ecvitem{Dates}{04/2005 -- 12/2005}
%\ecvitem{Institution}{Autonomous University of Santa Cruz, Bolivia}
%\ecvitem{Qualification awarded}{\emph{Postgraduate Diploma in
% Regional Leadership}}

\\
\ecvitem{Dates}{02/2001 -- 12/2005}
\ecvitem{Institution}{Autonomous University of Santa Cruz, Bolivia}
\ecvitem{Degree awarded}{\emph{Computer Engineer}}
%\\
% \ecvitem{Dates}{1989 -- 2000}
% \ecvitem{Institution}{State-Subsidised School ``Fe y Alegr\'{i}a La Merced'', Bolivia}
% \ecvitem{Title of qualification awarded}{\emph{High School Diploma}}


\ecvsection{Work experience}
% \ecvitem{Dates}{Add separate entries for each relevant post occupied, starting from the most recent. (Remove if not relevant).}
% \ecvitem{Occupation or position held}{\ldots}
% \ecvitem{Main activities and responsibilities}{\ldots}
% \ecvitem{Name and address of employer}{\ldots}
% \ecvitem{Type of business or sector}{\ldots}
\ecvitem{Dates}{02/2014 -- present}
\ecvitem{Occupation or position held}{Research Associate in Software Testing}
\ecvitem{Institution}{Department of Computer Science
  - The University of Sheffield, United Kingdom}
%\ecvitem{Type of business or sector}{Education}
\\
\ecvitem{Dates}{06/2013 -- 09/2013}
\ecvitem{Occupation or position held}{Software Developer}
\ecvitem{Institution}{Google Summer of Code and NASA Ames Research Center, Mountain View, USA}
\ecvitem{Supervisors}{Prof. Corina P\u{a}s\u{a}reanu and Prof. Willem Visser}

\\
\ecvitem{Dates}{09/2012 -- 12/2012}
\ecvitem{Occupation or position held}{Predoctoral Research Intern}
\ecvitem{Institution}{NASA Ames Research Center, Mountain View, CA USA}
\ecvitem{Host Researcher}{Prof. Corina P\u{a}s\u{a}reanu}
\\
\ecvitem{Dates}{10/2009 -- 12/2013}
\ecvitem{Occupation or position held}{Predoctoral Research Grant, awarded by the Spanish
  Ministry of Science and Education under the National Program for Training Human Resources}
\ecvitem{Institution}{Department of Computer Science
  - Technical University of Madrid, Spain}
%\ecvitem{Type of business or sector}{Education}
\\
\\
\ecvitem{Dates}{12/2007 -- 08/2009}
\ecvitem{Occupation or position held}{IT Trainee}
\ecvitem{Institution}{IT Department - Autonomous
  University of Madrid, Spain}
%\ecvitem{Type of business or sector}{Education}
\\
\ecvitem{Dates}{06/2006 -- 09/2007}
\ecvitem{Occupation or position held}{.NET Developer}
\ecvitem{Institution}{ABJ Consulting, Santa Cruz, Bolivia}
%\ecvitem{Type of business or sector}{Information and Communication}
\\
\ecvitem{Dates}{02/2006 -- 12/2006}
\ecvitem{Occupation or position held}{Mathematics Laboratory Assistant}
\ecvitem{Institution}{Autonomous University of Santa Cruz, Bolivia}
%\ecvitem{Type of business or sector}{Education}
\\
\ecvitem{Dates}{02/2004 -- 12/2005}
\ecvitem{Occupation or position held}{Teaching Assistant (Principles
  of Informatics, Compilers)}
\ecvitem{Institution}{Autonomous University of Santa Cruz, Bolivia}
%\ecvitem{Type of business or sector}{Education}
\\
\ecvitem{Dates}{02/2004 -- 07/2004}
\ecvitem{Occupation or position held}{Teaching Assistant (Computer
  Programming, Data Structures)}
\ecvitem{Institution}{University of Aquino in Santa Cruz, Bolivia}
%\ecvitem{Type of business or sector}{Education}

\ecvsection{Selected Publications With Abstracts}
\bibliographystyle{myplain}
\nobibliography{mybibtex}

\ecvitem[0pt]{1}{\bibentry{ICSE2017_CodeDefenders}.}
\ecvitem[10pt]{}{\textbf{Abstract:} Writing good software tests is
  difficult and not every developer's favorite occupation. Mutation
  testing aims to help by seeding artificial faults (mutants) that
  good tests should identify, and test generation tools help by
  providing automatically generated tests. However, mutation tools
  tend to produce huge numbers of mutants, many of which are trivial,
  redundant, or semantically equivalent to the original program;
  automated test generation tools tend to produce tests that achieve
  good code coverage, but are otherwise weak and have no clear
  purpose. In this paper, we present an approach based on gamification
  and crowdsourcing to produce better software tests and mutants: The
  Code Defenders web-based game lets teams of players compete over a
  program, where attackers try to create subtle mutants, which the
  defenders try to counter by writing strong tests. Experiments in
  controlled and crowdsourced scenarios reveal that writing tests as
  part of the game is more enjoyable, and that playing Code Defenders results
  in stronger test suites and mutants than those produced by automated
  tools.}

\ecvitem[0pt]{2}{\bibentry{ShamshiriJRFMA2015}.}
\ecvitem[10pt]{}{\textbf{Abstract:} Rather than tediously writing unit tests manually, tools can be used
  to generate them automatically --- sometimes even resulting in higher code
  coverage than manual testing. But how good are these tests
  at actually finding faults?
  %
  To answer this question, we applied three state-of-the-art unit test
  generation tools for Java (Randoop, EvoSuite, and Agitar) to the 357
  real faults in the Defects4J dataset and investigated how well the
  generated test suites perform at detecting these faults.
%
  Although the automatically generated test suites detected 55.7\% of the faults overall,
  only 19.9\% of all the individual test suites
  detected a fault.
%
  By studying the effectiveness and problems of the individual tools
  and the tests they generate, we derive insights to support the development of
  automated unit test generators that achieve a higher fault detection rate.
%
  These insights include 1)~improving the obtained code coverage
  so that faulty statements are executed in the first
  instance, 2)~improving the propagation of
  faulty program states to an observable output, coupled with the generation of more
  sensitive assertions,
  and 3)~improving the simulation of the execution environment to
  detect faults that are dependent
  on external factors such as date and time.}

\ecvitem[0pt]{3}{\bibentry{ShamshiriRFM15}.}
\ecvitem[10pt]{}{\textbf{Abstract:}  An important aim in software testing is constructing a test suite with
  high structural code coverage -- that is, ensuring that most if not all
  of the code under test has been executed by the test cases
  comprising the test suite.
%
  Several search-based techniques have proved
  successful at automatically generating tests that achieve
  high coverage.
%
  %% PM: not sure about the use of the word "theoretical" here,
  %% it implies there's been a proof, which there hasn't been of course
  % In spite of the well-known theoretical benefits of
  However, despite the well-established arguments behind using
  evolutionary
  %advanced
  search algorithms (e.g., genetic algorithms) in preference to random search, it remains
    an open question whether the benefits can actually
  be observed in practice when generating unit test suites for object-oriented classes.
 %
  In this paper, we report an empirical study on the effects of using
  evolutionary algorithms (including a genetic algorithm and chemical
  reaction optimization) to generate test suites, compared with generating
  test suites incrementally with random search.  We apply the
  EvoSuite unit test suite generator to 1,000 classes randomly
  selected from the SF110 corpus of open source projects.
%
  Surprisingly, the results show
  %% PM: we don't do overall significance tests just on classes
  %% where there were significant differences, so this may be confusing.
  %% I swapped out significance for "real"
  %  no significant
  little difference between the coverage achieved by test suites
  generated with evolutionary search compared to those generated using
  random search. A detailed analysis reveals that the evolutionary search covers more branches of the type where standard fitness
  functions provide guidance.
  %% PM: I think this is more of a side-point which breaks the flow
  %% and seems a bit thrown on top
  %, improvement which is magnified by the
  %use of an archive of solutions.
  In practice, however, we observed
  that the vast majority of branches do not provide any guidance to the search. These results
  suggest that, although evolutionary algorithms are more effective at
  covering complex branches, a random search may suffice to achieve high coverage of most object-oriented
  classes.}

\ecvitem[0pt]{4}{\bibentry{RojasCVFA15}.}
\ecvitem[10pt]{}{\textbf{Abstract:}   Automated test generation techniques typically aim at maximising
  coverage of well-established structural criteria such as statement
  or branch coverage. In practice, generating tests only for one
  specific criterion may not be sufficient when testing object
  oriented classes, as standard structural coverage criteria do not
  fully capture the properties developers may desire of their unit
  test suites. For example, covering a large number of statements
  could be easily achieved by just calling the {\tt main} method of a
  class; yet, a good unit test suite would consist of smaller unit
  tests invoking individual methods, and checking return values and
  states with test assertions. There are several different properties
  that test suites should exhibit, and a search-based test generator
  could easily be extended with additional fitness functions to
  capture these properties. %

  % However, does search-based testing scale to combinations of multiple
  % criteria, and do the resulting test suites grow in size
  % substantially?
  However, does search-based testing scale to combinations of multiple
  criteria, and what is the effect on the size and coverage of the
  resulting test suites?
%
  To answer these questions, we extended the EvoSuite unit test
  generation tool to support combinations of multiple test
  criteria, defined and implemented several different criteria, and
  applied combinations of criteria to a sample of 650 open source
  Java classes.
%
  Our experiments suggest that optimising for several criteria at the
  same time is feasible without increasing computational costs: When
  combining nine different criteria, we observed an average decrease
  of only 0.4\% for the constituent coverage criteria, while the test
  suites may grow up to 70\%.}
\ecvsection{Complete Research Publication Record}
%% \ecvitem{}{\ecvspace{-1cm}\textbf{2012}}

\ecvitem{\textbf{2017}}{}


\ecvitem[10pt]{}{\bibentry{SBST2017_LocalOptima}.}
\ecvitem[10pt]{}{\bibentry{SBST2017_EvoSuite}.}
\ecvitem[10pt]{}{\bibentry{ICSE2017_CodeDefenders}.}
\ecvitem[10pt]{}{\bibentry{ICSE_SEET2017_CodeDefenders}.}

\ecvitem{\textbf{2016}}{}

\ecvitem[10pt]{}{\bibentry{STVR_seeding}.}
\ecvitem[10pt]{}{\bibentry{emse16_effectiveness}.}
%%\ecvitem[10pt]{}{\bibentry{PPIG16_TeachingTesting}.}
\ecvitem[10pt]{}{\bibentry{ECSEE16_MutationEducation}.}
\ecvitem[10pt]{}{\bibentry{Mutation16_CodeDefenders}.}

\ecvitem{\textbf{2015}}{}
\ecvitem[10pt]{}{\bibentry{ShamshiriJRFMA2015}.}
\ecvitem[10pt]{}{\bibentry{ShamshiriRFM15}.}
\ecvitem[10pt]{}{\bibentry{RojasFA15}.}
\ecvitem[10pt]{}{\bibentry{RojasCVFA15}.}

\ecvitem{\textbf{2014}}{}
\ecvitem[10pt]{}{\bibentry{AlbertAGR14}.}
\ecvitem{\textbf{2013}}{}
\ecvitem[10pt]{}{\bibentry{AlbertGGZRS13}.}
\ecvitem[10pt]{}{\bibentry{2013-nasa}.}
\ecvitem{\textbf{2012}}{}
\ecvitem[10pt]{}{\bibentry{AlbertAACFGGMPRRZ12}.}
\ecvitem[10pt]{}{\bibentry{2013-guided}.}
\ecvitem[10pt]{}{\bibentry{2012-jms2abs}.}
\ecvitem{\textbf{2011}}{}
\ecvitem[10pt]{}{\bibentry{2012-ResourceTDG}.}
\ecvitem[10pt]{}{\bibentry{2011-NEPSonClusters}.}
\ecvitem{\textbf{2010}}{}
\ecvitem[10pt]{}{\bibentry{2011-CompTDG}.}
\ecvitem[10pt]{}{\bibentry{2010-JHSYS}.}
\ecvitem{\textbf{2009}}{}
\ecvitem[10pt]{}{\bibentry{2009-JNEPS}.}

\ecvsection{Academic Service}
\ecvitem{\textbf{Program Chair}}{MUTATION'17}

\ecvitem{\textbf{Program Committee Member}}{A-MOST'17, AST'17, GECCO'17, ICST'17
  (Tool Papers Track), ISSTA'17
  (AE), SBST'17, A-MOST'16, AST'16, CHESE'16, GECCO'16, ISSTA'16 (AE), MUTATION'16, ICST'15
  (Tool Track)}

\ecvitem{\textbf{Journal Reviewer}}{EMSE, JSS, SCICO, SCP, STVR, TSE}
\ecvitem{\textbf{Sub-reviewer}}{TAP'17, iFM'13, KDPD'13,
  NFM'13, TACAS'13, WFLP'13, SCAM'10}

\ecvsection{Research Activities}
\ecvitem{Activity}{Research visit}
\ecvitem{Institution}{University of Buenos Aires, Argentina}
\ecvitem{Host Researcher}{Dr. Juan Pablo Galeotti}
\ecvitem{Dates}{18--22/04/2016}
\\
\ecvitem{Activity}{Research visit}
\ecvitem{Institution}{Technical University of Madrid, Spain}
\ecvitem{Host Researcher}{Prof. Natalia Juristo}
\ecvitem{Dates}{14--18/03/2016}
\\
\ecvitem{Activity}{10th TAROT International Summer School on Training And Research On Testing}
\ecvitem{Organiser}{Facultade de Ingenharia, Universidade do Porto
  (Location: Porto, Portugal)}
%\ecvitem{Location}{Porto, Portugal}
\ecvitem{Dates}{29/06--04/07/2014}
\\
\ecvitem{Activity}{8th LASER Summer School on Soft.
  Eng.: Tools for Practical Software Verification}
\ecvitem{Organiser}{ETH Chair of Software Engineering (Location: Elba Island, Italy)}
%\ecvitem{Location}{Elba Island, Italy}
\ecvitem{Dates}{04--10/09/2011}

\\
\ecvitem{Activity}{Research visit}
\ecvitem{Institution}{KTH Royal Institute of Technology, Stockholm, Sweden}
\ecvitem{Host Researcher}{Prof. Mads Dam}
\ecvitem{Dates}{05/2011 -- 07/2011}

\\
\ecvsection{Software Projects}

%\ecvitem{\textbf{Current Projects}}{}
%\\
\ecvitem{Name}{Code Defenders}
\ecvitem{Description}{Mutation testing game for software testing
  education and crowdsourcing}
\ecvitem{URL}{\url{code-defenders.org}}
\ecvitem{Role}{Lead developer}
\ecvitem{}{}
\ecvitem{Name}{EvoSuite}
\ecvitem{Description}{Search-based unit test generation for Java}
\ecvitem{URL}{\url{evosuite.org}}
\ecvitem{Role}{Developer}
\\
\ecvitem{\textbf{Past Projects}}{}
\\
\ecvitem{Name}{PET}
\ecvitem{Description}{Test data generation using symbolic execution}
\ecvitem{URL}{\url{costa.ls.fi.upm.es/pet}}
\ecvitem{Role}{Developer}

\ecvsection{Research Projects}


%\ecvitem{\textbf{Current Project}}{}

\ecvitem{Project title}{GReaTest: \emph{Growing Readable Software Tests}}
\ecvitem{Financing Organisation}{Engineering and Physical Sciences
  Research Council  (EPSRC) (EP/N023978/1)}
\ecvitem{Participant Organisations}{University of
  Sheffield, Barclays Bank Plc, Google, Microsoft}
\ecvitem{Duration}{03/2016 -- 02/2020}
\ecvitem{Main Researcher}{Dr. Gordon Fraser}

\\
\ecvitem{\textbf{Past Projects}}{}

\ecvitem{Project title}{EXOGEN: \emph{Explorative Test Oracle Generation}}
\ecvitem{Financing Organisation}{Engineering and Physical Sciences
  Research Council  (EPSRC) (EP/K030353/1)}
\ecvitem{Participant Organisations}{University of Sheffield, Google, Microsoft}
\ecvitem{Duration}{02/2014 -- 08/2015}
\ecvitem{Main Researcher}{Dr. Gordon Fraser}

\\
\ecvitem{Project title}{PROMETIDOS-CM: \emph{Madrid Program in Rigorous Methods for the Development of Software}}
\ecvitem{Financing Organisation}{Madrid Regional Government (CAM S2009TIC-1465)}
\ecvitem{Participant Organisations}{IMDEA Software (Spain), Technical University of Madrid (Spain), Universidad Complutense de Madrid (Spain)}
\ecvitem{Duration}{January 2010 -- December 2013}
%\ecvitem{Amount}{\EUR{165,049}}
\ecvitem{Main Researcher}{Prof. Francisco Bueno}

\\
\ecvitem{Project title}{DOVES: \emph{Development Of Verifiable and
    Efficient Software}}
\ecvitem{Financing Organisation}{Spanish Ministry of Science
  and Innovation (TIN 2008-05624)}
\ecvitem{Participant Organisations}{Technical University of Madrid (Spain)}
\ecvitem{Duration}{January 2009 -- December 2013}
%\ecvitem{Amount}{\EUR{368,000} + 2 FPI, 1 Technician}
\ecvitem{Main Researcher}{Prof. Manuel Hermenegildo}

\\
\ecvitem{Project title}{HATS: \emph{Highly Adaptable and Trustworthy Software using Formal
  Models}}
\ecvitem{Financing Organisation}{CE ICT GA\#231620}
\ecvitem{Participant Organisations}{CTH (Sweden),
UIO (Norway),
KTH (Sweden),
Technical University of Madrid (Spain),
UKL (Germany),
BOL (Italy),
CWI (Holland),
NRS (Norway),
FRH (Germany),
FRG (Holland),
KUL (Belgium)}
\ecvitem{Duration}{March 2009 -- February 2013}
%\ecvitem{Amount}{\EUR{478,437}}
\ecvitem{Main Researcher}{Prof. Germ\'{a}n Puebla}



%\ecvitem{}{}
%\ecvitem{Summer School}{1st PROMETIDOS-CM Summer School}
%\ecvitem{Organiser}{Complutense University of Madrid}
%\ecvitem{Location}{Madrid, Spain}
%\ecvitem{Dates}{19--21/09/2011}

%% \ecvsection{Personal skills and~competences}
\ecvsection{Languages}

\ecvmothertongue[5pt]{Spanish}
\ecvitem{\large Proficient}{English}
%\ecvlanguageheader{(*)}
%\ecvlastlanguage{English}{\ecvCTwo}{\ecvCTwo}{\ecvCTwo}{\ecvCTwo}{\ecvCTwo}
%\ecvlastlanguage{German}{\ecvAOne}{\ecvAOne}{\ecvAOne}{\ecvAOne}{\ecvAOne}
%\ecvlanguagefooter[10pt]{(*)}

% \ecvitem[10pt]{\large Social skills and competences}{Replace this text by a description of these competences and indicate where they were acquired (remove if not relevant).}
% \ecvitem[10pt]{\large Organisational skills and competences}{Replace this text by a description of these competences and indicate where they were acquired (remove if not relevant).}
% \ecvitem[10pt]{\large Technical skills and competences}{Replace this text by a description of these competences and indicate where they were acquired (remove if not relevant).}
% \ecvitem[10pt]{\large Computer skills and competences}{Replace this text by a description of these competences and indicate where they were acquired (remove if not relevant).}
% \ecvitem[10pt]{\large Artistic skills and competences}{Replace this text by a description of these competences and indicate where they were acquired (remove if not relevant).}
% \ecvitem[10pt]{\large Other skills and competences}{Replace this text by a description of these competences and indicate where they were acquired (remove if not relevant).}
% \ecvitem{\large Driving licence(s)}{State here whether you hold a driving licence and if so for which categories of vehicle. (Remove if not relevant).}

% \ecvsection{Additional information}
% \ecvitem[10pt]{}{Include here any other information that may be relevant, for example contact persons, references, etc. (Remove heading if not relevant).}

% \ecvitem{}{\textbf{Personal interests}}
% \ecvitem{}{\ldots}

% \ecvsection{Annexes}
% \ecvitem{}{List any item attached to the CV}
\end{europecv}


\end{document} 
